\section*{VLAN}
(von K. Lapp)\\

VLANs = Virtuelle Netze \\
\cite[S.167-169]{zisler2018computer} \\
\cite{bsiLogSeg}\\

VLANs dienen der logischen Segmentierung von Netzen. Es sind logische Teilnetze, die an Switches gebildet werden. Es können Gruppen gebildet werden, ohne dass in die physische Vernetzung eingegriffen wird.
Gründe für den VLAN Einsatz nach \cite[S.167]{zisler2018computer}:
\begin{itemize}
  \item Eindämmung von Broadcast durch mehrere Broadcast Domänen
  \item Abbildung der betrieblichen Organisationsstruktur (Abteilungen)
  \item Einteilung des Netzes nach Anwendung 
\end{itemize}



Arten von VLANs: 
\begin{enumerate}
  \item Statische VLANs = Portbasierte VLANs: Switch-Ports werden fest einem VLAN zugeordnet. Port kann nur einem VLAN zugeordnet werden. 
  \item Dynamisches VLAN = Paketbasiertes VLAN, auch tagged VLAN: Ein Port kann mehreren VLANs angehören. Pakete werden gekennzeichnet welchem VLAN sie angehören. Achtung hohe Manipulationsgefahr. 
\end{enumerate}

