\section{BSI - Bundesamt für Sicherheit in der Informationstechnik}\index{BSI}\index{Bundesamt für Sicherheit in der Informationstechnik}
(von M. Alpers)\\

Das BSI befasst sich laut eigener Angabe \glqq{}mit allen Fragen rund um die IT-Sicherheit in der Informationsgesellschaft.\grqq{} (Quelle: \url{https://www.bsi.bund.de/DE/Service/FAQ/faq_node.html})\\

Es bietet folgende für diese Arbeit relevanten Dienste:\\
(Quelle: \url{https://www.bsi.bund.de/DE/Service/FAQ/faq_node.html})

\begin{itemize}
	\item \glqq{}Prüfung, Zertifizierung und Akkreditierung von IT-Produkten und -Dienstleistungen\grqq{}
	\item \glqq{}Warnung vor Schadprogrammen oder Sicherheitslücken in IT-Produkten und -Dienstleistungen\grqq{}
	\item \glqq{}Entwicklung einheitlicher und verbindlicher IT-Sicherheitsstandards\grqq{}
\end{itemize}

\subsection{Cybersicherheitsstrategie 2016}\index{Cybersicherheitsstrategie}

Das Bundesministerium des Inneren hat als Grundlage staatlicher Aktivitäten im Bereich IT-Sicherheit die sog. \emph{Cybersicherheitsstrategie} veröffentlicht. (Letzte Aktualisierung: 2016) Darin wird es als Aufgabe des Bundes definiert, \glqq{}Unternehmen in Deutschland zu schützen\grqq{} (Quelle: \url{https://www.bmi.bund.de/SharedDocs/downloads/DE/publikationen/themen/it-digitalpolitik/cybersicherheitsstrategie-2016.pdf} : \emph{Handlungsfeld 2} (S. 10)) Die konkrete Umsetzung dieses Ziels wird jedoch nicht ausgeführt, sondern lediglich die Absicht bekundet, stärken mit entsprechenden Ansprechstellen zusammen zu arbeiten. (Gleiche Quelle, S. 22)\\

\subsection{Überblickspapiere des BSI}

Unter der Bezeichnung \emph{Überblickspapiere}\index{Überblickspapiere} veröffentlicht das BSI Informationen über spezifische Sicherheitsbedrohungen bei der Nutzung bestimmter Systeme. Es liegen zurzeit die folgenden Überblickspapiere vor:\\
(Quelle: \url{https://www.bsi.bund.de/DE/Themen/ITGrundschutz/Ueberblickspapiere/Ueberblickspapiere_node.html})

\begin{itemize}
	\item Android (Stand: 20.11.2014, Android v 4.4.4)
	\item Apple iOS (Stand: 24.07.2013, iOS v 6)
	\item Online-Speicher (Stand: 08.11.2012)
	\item Smartphones (Stand: Unbekannt)
	\item Netzzugangskontrolle (Stand: Unbekannt)\\
	Hier wird der Umgang mit fremden mobilen Geräten in einer geschützten Umgebung betrachtet.
	\item IT-Consumerisation und BYOD (Stand: 31.07.13)\\
	Hier wird der Umgang mit mobilen Endgeräten innerhalb eines Unternehmens betrachtet, die von Mitarbeitern mitgebracht werden.
\end{itemize}

\subsubsection{Fazit:}

\begin{itemize}
	\item Die Überblickspapiere zu Android und iOS sind mit vier bzw. fünf Jahren veraltet, da sie vorrangig auf die konkreten Gefahren der jeweiligen OS-Version eingehen.
	\item Das Überblickpapier Consumerisation und BYOD ist zwar mehr als vier Jahre alt, aber allgemeiner formuliert, sodass es sich als Grundlage einer Sicherheitsstrategie für ein Unternehmen nutzen lässt.
	\item Gleiches gilt für das Überblickpapier Online-Speicher (Cloud), Smartphones
	\item Das Überblickpapier Netzzugangskontrolle ist nicht datiert, geht zum Teil auf konkrete Standards aber auch auf grundlegende Sicherheitszenarien ein. Es sollte damit zumindest als Grundlage für den Entwurf eines Sicherheitskonzepts taugen.
\end{itemize}

\subsection{BSI und Unternehmen}

Auch für kleine und mittlere Unternehmen steht das BSI als Ansprechpartner zur Verfügung. Darüber hinaus wird auf die Zusammenarbeit mit der Allianz für Cybersicherheit (Intiative der BITCOM) verwiesen.

\subsection{Onlineschulung für IT-Grundschutz}

Das BSI bietet eine kostenlose Onlineschulung zum modernisierten IT-Grundschutz zur Verfügung: \url{https://www.bsi.bund.de/DE/Themen/ITGrundschutz/ITGrundschutzSchulung/itgrundschutzschulung_node.html}