\section*{Datenschutz und Jugendschutz}

\subsection{Blog von Microsoft: Forderung nach Regulierung von Gesichtserkennung}
(von M. Alpers, muss noch ausgearbeitet werden)\\

Unter \url{https://blogs.microsoft.com/on-the-issues/2018/07/13/facial-recognition-technology-the-need-for-public-regulation-and-corporate-responsibility/} findet sich ein Blog-Eintrag des Microsoft-Präsidenten Brad Smith (weitere Blogeinträge von ihm unter: \url{https://blogs.microsoft.com/on-the-issues/author/bradsmith/} ), in dem dieser Politik und Wirtschaft zum verantwortlichen Umgang mit Gesichtserkennung aufruft.

\subsection{WhatsApp datenschutzkonform nutzen}
(von M. Alpers)\\

Ein zentrales Problem für den Datenschutz (auch) in Unternehmen sind Funktionen zum \emph{Data Mining}\index{Data Mining} in Anwendungen, die zum Teil als sog. \emph{Bloatware}\index{Bloatware} installiert werden. Deren Geschäftsmodell basiert dann nicht darauf, dass der Nutzer eine Lizenzgebühr zahlt, sondern durch die Offenlegung eines Teils seiner Daten die Grundlage für das Wirtschaftmodell des Unternehmens bildet. Im Falle von WhatsApp sind die offengelegten Daten unter anderen die im Smartphone gespeicherten Kontakte des Nutzers. Dies widerspricht der DSVGO, da diese ein ausdrückliches Einverständnis der Betroffenen fordert.\\

\glqq{}Das Unternhemen  Backes SRT hat eine App namens \emph{WhatsBox} veröffentlicht,\grqq{} die nun verspricht, WhatsApp in einer Sandbox zu betreiben und dadurch WhatsApp DSVGO-konform nutzbar zu machen. Über die lassen sich Kontakte explizit für WhatsApp freigeben.\\

\url{https://www.backes-srt.com/de/loesungen/whatsbox/}