\section*{Datenschutz und Jugendschutz}

\subsection{Blog von Microsoft: Forderung nach Regulierung von Gesichtserkennung}
(von M. Alpers, muss noch ausgearbeitet werden)\\

Unter \url{https://blogs.microsoft.com/on-the-issues/2018/07/13/facial-recognition-technology-the-need-for-public-regulation-and-corporate-responsibility/} findet sich ein Blog-Eintrag des Microsoft-Präsidenten Brad Smith (weitere Blogeinträge von ihm unter: \url{https://blogs.microsoft.com/on-the-issues/author/bradsmith/} ), in dem dieser Politik und Wirtschaft zum verantwortlichen Umgang mit Gesichtserkennung aufruft.

\subsection{WhatsApp datenschutzkonform nutzen}
(von M. Alpers)\\

Ein zentrales Problem für den Datenschutz (auch) in Unternehmen sind Funktionen zum \emph{Data Mining}\index{Data Mining} in Anwendungen, die zum Teil als sog. \emph{Bloatware}\index{Bloatware} installiert werden. Deren Geschäftsmodell basiert dann nicht darauf, dass der Nutzer eine Lizenzgebühr zahlt, sondern durch die Offenlegung eines Teils seiner Daten die Grundlage für das Wirtschaftmodell des Unternehmens bildet. Im Falle von WhatsApp sind die offengelegten Daten unter anderen die im Smartphone gespeicherten Kontakte des Nutzers. Dies widerspricht der DSVGO, da diese ein ausdrückliches Einverständnis der Betroffenen fordert.\\

\glqq{}Das Unternhemen  Backes SRT hat eine App namens \emph{WhatsBox} veröffentlicht,\grqq{} die nun verspricht, WhatsApp in einer Sandbox zu betreiben und dadurch WhatsApp DSVGO-konform nutzbar zu machen. Über die lassen sich Kontakte explizit für WhatsApp freigeben.\\

\url{https://www.backes-srt.com/de/loesungen/whatsbox/}


\subsection{Effizientes und verantwortungsvolles
Datenmanagement im
Zeitalter der DSGVO}
(KL)\\

Quelle: \cite{Brockmann2018}
Nach DSGVO müssen Unternehmen jederzeit: 
\begin{itemize}
\item	darlegen können, welche personenbezogenen Daten in welchen
Systemen für welchen Zweck auf welcher rechtlichen Basis von
wem gespeichert und verarbeitet werden,
\item darlegen können, wer diese Daten wie und wofür verarbeitet
und nutzt,
\item Anfragen binnen eines Monats vollständig, lesbar und eindeutig
beantworten,
\item Änderungs-, Widerruf- oder Löschungswuünsche seitens der
betroffenen Person auf allen Systemen im Unternehmen umsetzen,
\item sicherstellen, dass (auch externe) Datenverarbeiter der DSGVO
nachkommen und
\item die Konformität mit den datenschutzrechtlichen Anforderungen
prüfen, nachbessern und nachweisen.

\end{itemize}

in Unternehmen herrschen in der Regel heterogene Datenlandschaften vor, die eine DSGVO-konformität erschweren.

Der Artikel versucht dem Problem durch semantische Technologien zu begegnen (Ontologien/Vokabularen). Bsp Google Knowledgegraph.
Die semantische Technologie kann als Enterprise Knowledge
Graph (EKG) auch auf Unternehmen und Institutionen angewendet
werden.
Kern der EKG ist das standardisierte, Statement-zentrische
RDF-Datenmodell. Dieses Modell strukturiert Daten nach einer
Subjekt-Prädikat-Objekt-Syntax, wodurch Daten jeglicher Art
untereinander referenziert und verlinkt werden können.
Bei einigen EKG-Ansätzen wird die Transformation aller Unternehmensdaten
in RDF empfohlen. Dies hat in der Vergangenheit
jedoch zu Problemen bei der Skalierbarkeit geführt, da alle
Daten transformiert werden mussten. Als Alternative hat sich
deshalb das rein Metadaten-basierte Management der Quelldaten
mittels eines semantischen Datenkatalogs herausgebildet. Hierbei
bleiben die Quelldaten in ihren Applikationen unberührt,
und nur ihre Metadaten (im RDF-Format) werden in einem zentralen
Datenkatalog zusammengeführt und mit der Quelle verlinkt.
Insbesondere für die Umsetzung der DSGVO ist die Katalogisierung,
Beschreibung und Vernetzung von persönlichen Daten
aus einer Vielzahl von Quellen eine essentielle Voraussetzung.
In der Praxis wird in die bestehende Datenlandschaft eines
Unternehmens ein semantischer Datenkatalog integriert. In diesem
werden die Metadaten, als das Wissen über die im Unternehmen
vorliegenden Daten zusammengeführt und gespeichert.
Die Quelldaten in den verteilten Applikationen im Unternehmen
werden dabei nicht berührt.
