\documentclass[11pt, a4paper, twoside]{book}

\usepackage{palatino, url}
\usepackage[ngermanb]{babel}
\usepackage[utf8]{inputenc}
\setlength{\parindent}{0cm}

\usepackage{amsfonts}	% für mathematische Symbole wie Menge der nat. Zahlen
\usepackage{amsmath}
\usepackage{textcomp}	% für \textmu also das Mü- bzw. Mikro-Symbol
\usepackage{makeidx}	% für ein "schöneres" Stichwortverzeichnis
\makeindex				% erzeugt das Stichwortvereichnis, blendet es jedoch nicht hier ein. Dafür am Ende den entspr. Code verwenden.

\bibliographystyle{plain}

\begin{document}
	
	%%% Title declaration begin %%%
	
	\title{Ausarbeitung zu \\ 01909 - Seminar IT-Sicherheit \\ Maßnahmen zur Absicherung von privaten und kleinen Unternehmensnetzwerken}
	\author{B.Sc. Markus Alpers, 6872530 \\ Dipl. -Met. Kerstin Lapp, 5105200}
	\date{\today}
	
	\maketitle
	
	%%% Title declaration end %%%
	
	\tableofcontents
	
	%%% Chapters %%%


	
	%%% Appendices %%%
	
	\renewcommand{\indexname}{Stichwortverzeichnis}		% Legt den Titel des Stichwortverzeichnisses fest.
	\addcontentsline{toc}{chapter}{Stichwortverzeichnis}
	\printindex
	\nocite{*}
	\bibliography{../lit.bib}

\end{document}