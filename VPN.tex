\section*{VPN Virtuelles privates Netzwerk}
\cite[S. 372-378]{zisler2018computer}
\begin{itemize}
  \item Verbindung zweier Teilnetze oder
  \item Verbindung eines Einzelrechners mit einem LAN/Intranet
\end{itemize}
\\

\cite{lipp2007vpn}
\begin{itemize}
  \item Ein Netzwerk, das ein anderes,öffentliches Netzwerk benutzt, um privte Daten zu transportieren. 
  \item EIn VPN trennt den Transport privater Datenpakete oder privater Datenframes von anderen, es bietet nur die Sicherheit, dass die Pakete nicht zu falschen Empfänger geleitet werden. Weitere Sicherheitsmaßnahmen sind optional
  \item Gründe für VPN: \begin{itemize}
  						\item Veränderung der Geschäftsprozesse (B2B, B2C)
  						\item Dezentralisierung, Globalisirung
  						\item Veränderung der Wettbewerbssituation
  						\item Mobilität und Flexibilität
  						\item Kostenoprimierung
  						\item Sicherheit
						\end{itemize}
						
	\item VPN-Typen: \begin{itemize}	
					 \item Remote Access VPN : Verbindet Einzelrechner mit dem Intranet
					 \item Branch office VPN = site to site VPN: Verbindet verschiedene Intranets miteinander.
					 \item Extranet VPN: öffnet das private Netz auch für externe Personen => Datenpakete müssen gesondert behandelt werden
					 \end{itemize}	
   \item Anforderungen an die VPN Sicherheit: (Kapitel 2)
					\begin{itemize}
					\item Datenvertraulichkeit => Verschlüsselungsverfahren
					\item Schlüsselmanagement: Schlüselerzeugung, Integritätsprüfung, Authentifizierung, Schlüsselverteilung.
					\item Paketauthentifizierung: Jedes Paket muss Authentifiziert werden, dass es tatsächlich vom Absender stammt. 
					\item Datenintigrität: Wurde Paket während des Transports verändert?
					\item Benutzerauthentifizierung: Wichtig beim Remote Access VPN. Nutzer muss Identität zuverlässig nachweisen.
					\item Benutzerauthorisierung: Vor allem bei Extranet, Aufgabe der Betriebssysteme.
					\item Schutz vor Sabotage
					\item Schutz vor unerlaubtem Eindringen
					\end{itemize}
  \item Sicherheitstechnologien: (Kapitel 4)					
 
\end{itemize}
