\section*{VPN Virtuelles privates Netzwerk}
\cite[S. 372-378]{zisler2018computer}
\begin{itemize}
  \item Verbindung zweier Teilnetze oder
  \item Verbindung eines Einzelrechners mit einem LAN/Intranet
\end{itemize}

\cite{lipp2007vpn}
\begin{itemize}
  \item Ein Netzwerk, das ein anderes,öffentliches Netzwerk benutzt, um privte Daten zu transportieren. 
  \item EIn VPN trennt den Transport privater Datenpakete oder privater Datenframes von anderen, es bietet nur die Sicherheit, dass die Pakete nicht zu falschen Empfänger geleitet werden. Weitere Sicherheitsmaßnahmen sind optional
  \item Gründe für VPN: \begin{itemize}
  						\item Veränderung der Geschäftsprozesse (B2B, B2C)
  						\item Dezentralisierung, Globalisirung
  						\item Veränderung der Wettbewerbssituation
  						\item Mobilität und Flexibilität
  						\item Kostenoprimierung
  						\item Sicherheit
						\end{itemize}
						
	\item VPN-Typen: \begin{itemize}	
					 \item Remote Access VPN : Verbindet Einzelrechner mit dem Intranet
					 \item Branch office VPN = site to site VPN: Verbindet verschiedene Intranets miteinander.
					 \item Extranet VPN: öffnet das private Netz auch für externe Personen => Datenpakete müssen gesondert behandelt werden
					 \end{itemize}	
   \item Anforderungen an die VPN Sicherheit: (Kapitel 2)
					\begin{itemize}
					\item Datenvertraulichkeit $\rightarrow$ Verschlüsselungsverfahren
					\item Schlüsselmanagement: Schlüselerzeugung, Integritätsprüfung, Authentifizierung, Schlüsselverteilung.
					\item Paketauthentifizierung: Jedes Paket muss Authentifiziert werden, dass es tatsächlich vom Absender stammt. 
					\item Datenintigrität: Wurde Paket während des Transports verändert?
					\item Benutzerauthentifizierung: Wichtig beim Remote Access VPN. Nutzer muss Identität zuverlässig nachweisen.
					\item Benutzerauthorisierung: Vor allem bei Extranet, Aufgabe der Betriebssysteme.
					\item Schutz vor Sabotage
					\item Schutz vor unerlaubtem Eindringen
					\end{itemize}
  \item Sicherheitstechnologien: (Kapitel 4): 
  					\begin{itemize}
  					\item Datenanalyse(Allgemein zur Datensicherheit, nicht nur VPN betreffend): Welche Daten müssen wie lange und vor wem gesichert werden. Wie sichd verletzungen dieser Sicherheut zu bewerten, zu welchem Preis dürfen die Sicherheitsziele erreicht werden. 
  					\item Die Sicherheitsanforderungen an ein VPN leiten sich aus der Security Policy (Sicherheitsrichtlinie) eines Unternehmens ab. 
  					\item Vom Standpunkt der Sicherheit wäre eine Ende zu Ende Verschlüsselung der Daten auf Applikationsebene die beste Methode. ABER technisch und organisatorisch nicht handhabbar(stand 2007).
  					\item Verbreitete Lösung: Sicherheit auf Netzwerkebene: OSI Schicht 3
  					\item Für ein Internet VPN ist Verschlüsselung auf Ebene des IP-Protokolls das Sicherste und Sinnvollste. Weil bei Verschlüsselung in den Schichten 1 und 2 lägen die Daten an den Vermittlungssystemen im Klartext vor, da sie in der Schicht 3 verarbeitet werden. 
  					\item Sicherheit in der Netzwerkschicht mit IP-Securitiy (IPSec). IPSec : Standard für Sicherheit auf IP Ebene, darin sind und dem eng verbundenen IKE-Protokoll (Internet Key exchange) sind verschiedene Verschlüsselungs- und Authentifizierungsverfahren sowie Schlüsselaustausch- und Schlüsselverwaltungsprotokolle festgelegt, die hohe Interoperabilität gewährleisten. 
  					\item Sicherheit auf der Transportschicht mit SSl und TSL
  					\item Kapitel 4: IPSec
  					\item Kapitel 5: IKE
  					\item Kapitel 6: SSL
  					\end{itemize}					
 
\end{itemize}


Sicherheit im Internet 2 1867 KE3 S 91 -113: 

Motivation:
\begin{itemize}
	\item Firmennetze verschiedener Standorte wie ein einziges großes lokales Netz : Intranet / Site to Site. Lediglich auf dem Weg durch das Verbindende Netz werden die Daten verschlüsselt.
	\item Transparenter Zugriff auf das lokale Netz für Mobile Mitarbeiter: remote Access. Meist über das Internet. End to site. Vor allem bei Zugang über WLAN sollte das mobile Endgerät den Datenverkehr verschlüsseln  
\end{itemize}
