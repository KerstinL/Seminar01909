\documentclass[11pt, a4paper, twoside]{book}

\usepackage{palatino, url}
\usepackage[ngermanb]{babel}
\usepackage[utf8]{inputenc}
\setlength{\parindent}{0cm}

\usepackage{amsfonts}	% für mathematische Symbole wie Menge der nat. Zahlen
\usepackage{amsmath}
\usepackage{textcomp}	% für \textmu also das Mü- bzw. Mikro-Symbol
\usepackage{makeidx}	% für ein "schöneres" Stichwortverzeichnis
\makeindex				% erzeugt das Stichwortvereichnis, blendet es jedoch nicht hier ein. Dafür am Ende den entspr. Code verwenden.

\bibliographystyle{plain}

\begin{document}
	
	%%% Title declaration begin %%%
	
	\title{Notizen zu \\ 01909 - Seminar IT-Sicherheit \\ Maßnahmen zur Absicherung von privaten und kleinen Unternehmensnetzwerken}
	\author{B.Sc. Markus Alpers, 6872530 \\ Kerstin Lapp, 515200}
	\date{\today}
	
	\maketitle
	
	%%% Title declaration end %%%
	
	\tableofcontents
	
	%%% Chapters %%%
	
	\chapter{Begriffe}

	\section*{Schutzziele und Bedrohungen}

Angriffsziele (nach 1866 KE1 S.11):

\begin{itemize}
  \item Kommunikationswege
  \item Computer
  \item Daten
\end{itemize}
Schutzziele (nach 186 KE1 S.12-16):

\begin{itemize}
  \item Vertraulichkeit: Daten sind nur befugten Personen zugänglich. \\
  Bedrohung: unbefugter Informationsgewinn.
  \item Integrität: Daten sind korrekt und unverändert. \\ Bedrohung: Unbefugte Modifikation
  \item Authentizität: Daten stammen von vorgeblichen Erzeuger. \\
  Bedrohung: unbefugte Erzeugung
  \item Verfügbarkeit: Daten können von befugten Personen gelesen/bearbeitet werden. \\
  	Bedrohung: unbefugte Unterbrechung
\end{itemize}


	\section*{VLAN}

\cite[S.167-169]{zisler2018computer} \\

	\section*{VPN Virtuelles privates Netzwerk}
\cite[S. 372-378]{zisler2018computer}
\begin{itemize}
  \item Verbindung zweier Teilnetze oder
  \item Verbindung eines Einzelrechners mit einem LAN/Intranet
\end{itemize}
\\

\cite{lipp2007vpn}
\begin{itemize}
  \item Ein Netzwerk, das ein anderes,öffentliches Netzwerk benutzt, um privte Daten zu transportieren. 
  \item EIn VPN trennt den Transport privater Datenpakete oder privater Datenframes von anderen, es bietet nur die Sicherheit, dass die Pakete nicht zu falschen Empfänger geleitet werden. Weitere Sicherheitsmaßnahmen sind optional
  \item Gründe für VPN: \begin{itemize}
  						\item Veränderung der Geschäftsprozesse (B2B, B2C)
  						\item Dezentralisierung, Globalisirung
  						\item Veränderung der Wettbewerbssituation
  						\item Mobilität und Flexibilität
  						\item Kostenoprimierung
  						\item Sicherheit
						\end{itemize}
						
	\item VPN-Typen: \begin{itemize}	
					 \item Remote Access VPN : Verbindet Einzelrechner mit dem Intranet
					 \item Branch office VPN = site to site VPN: Verbindet verschiedene Intranets miteinander.
					 \item Extranet VPN: öffnet das private Netz auch für externe Personen => Datenpakete müssen gesondert behandelt werden
					 \end{itemize}	
   \item Anforderungen an die VPN Sicherheit: (Kapitel 2)
					\begin{itemize}
					\item Datenvertraulichkeit $\rightarrow$ Verschlüsselungsverfahren
					\item Schlüsselmanagement: Schlüselerzeugung, Integritätsprüfung, Authentifizierung, Schlüsselverteilung.
					\item Paketauthentifizierung: Jedes Paket muss Authentifiziert werden, dass es tatsächlich vom Absender stammt. 
					\item Datenintigrität: Wurde Paket während des Transports verändert?
					\item Benutzerauthentifizierung: Wichtig beim Remote Access VPN. Nutzer muss Identität zuverlässig nachweisen.
					\item Benutzerauthorisierung: Vor allem bei Extranet, Aufgabe der Betriebssysteme.
					\item Schutz vor Sabotage
					\item Schutz vor unerlaubtem Eindringen
					\end{itemize}
  \item Sicherheitstechnologien: (Kapitel 4): 
  					\begin{itemize}
  					\item Datenanalyse(Allgemein zur Datensicherheit, nicht nur VPN betreffend): Welche Daten müssen wie lange und vor wem gesichert werden. Wie sichd verletzungen dieser Sicherheut zu bewerten, zu welchem Preis dürfen die Sicherheitsziele erreicht werden. 
  					\item Die Sicherheitsanforderungen an ein VPN leiten sich aus der Security Policy (Sicherheitsrichtlinie) eines Unternehmens ab. 
  					\item Vom Standpunkt der Sicherheit wäre eine Ende zu Ende Verschlüsselung der Daten auf Applikationsebene die beste Methode. ABER technisch und organisatorisch nicht handhabbar(stand 2007).
  					\item Verbreitete Lösung: Sicherheit auf Netzwerkebene: OSI Schicht 3
  					\item Für ein Internet VPN ist Verschlüsselung auf Ebenedes IP-Protokolls das Sicherste und Sinnvollste. Weil bei Verschlüsselung in den Schichten 1 und 2 lägen die Daten an den Vermittlungssystemen im Klartext vor, da sie in der Schicht 3 verarbeitet werden. 
  					\end{itemize}					
 
\end{itemize}

	\section*{Firewall/DMZ}

Kurs 1866 KE4  S199: 

Screened Subnet/DMZ: Eigenes Teilnetz zwischen Intranet und Internet, welches an beiden Eden durch Paketfilter geschützt wird. \\
Architektur: Internet-Router-ALG-Rechner-Router-Intranet.\\
Router auf Seite des Internet lässt nur Pakete zum Rechner innerhalb der DMZ durch. Router zum Intranet lässt nur Pakete vom Rechner aus der DMZ durch. Die andere Richtung analog. Angreifer muss also mehrere Systeme angreifen ehe er das Intranet erreichen kann. ALG filtert zusätzlich Kommunikation auf der Anwendungsebene. Bei erhöhtem Arbeitsaufkommen sind verschiedene ALG für verschiedene Anwendungen möglich (ftp, telnet, http). Öffentliche Webserver sollten sich auch in der DMZ befinden.\\
DMZ kann auch mit einem Router mit 3 Anschlüssen verwirklicht werden: \begin{enumerate}
  \item Internet
  \item DMZ
  \item Intranet
\end{enumerate}
mit entsprechend konfiguriertem Paketfilter.

	\section*{Sicherheitsleitlinie/Polocies}
Siehe 1866 KE4 Kapitel 4.4 ab S.209 und BSI Grundschutzkatalog

Es müssen Richtlinien existieren WER fürdie Sicherheit verantwortlich ist (\cite{lipp2007vpn}).
	\section*{Datenschutz und Jugendschutz}

\subsection{Blog von Microsoft: Forderung nach Regulierung von Gesichtserkennung}
(von M. Alpers, muss noch ausgearbeitet werden)\\

Unter \url{https://blogs.microsoft.com/on-the-issues/2018/07/13/facial-recognition-technology-the-need-for-public-regulation-and-corporate-responsibility/} findet sich ein Blog-Eintrag des Microsoft-Präsidenten Brad Smith (weitere Blogeinträge von ihm unter: \url{https://blogs.microsoft.com/on-the-issues/author/bradsmith/} ), in dem dieser Politik und Wirtschaft zum verantwortlichen Umgang mit Gesichtserkennung aufruft.

\subsection{WhatsApp datenschutzkonform nutzen}
(von M. Alpers)\\

Ein zentrales Problem für den Datenschutz (auch) in Unternehmen sind Funktionen zum \emph{Data Mining}\index{Data Mining} in Anwendungen, die zum Teil als sog. \emph{Bloatware}\index{Bloatware} installiert werden. Deren Geschäftsmodell basiert dann nicht darauf, dass der Nutzer eine Lizenzgebühr zahlt, sondern durch die Offenlegung eines Teils seiner Daten die Grundlage für das Wirtschaftmodell des Unternehmens bildet. Im Falle von WhatsApp sind die offengelegten Daten unter anderen die im Smartphone gespeicherten Kontakte des Nutzers. Dies widerspricht der DSVGO, da diese ein ausdrückliches Einverständnis der Betroffenen fordert.\\

\glqq{}Das Unternhemen  Backes SRT hat eine App namens \emph{WhatsBox} veröffentlicht,\grqq{} die nun verspricht, WhatsApp in einer Sandbox zu betreiben und dadurch WhatsApp DSVGO-konform nutzbar zu machen. Über die lassen sich Kontakte explizit für WhatsApp freigeben.\\

\url{https://www.backes-srt.com/de/loesungen/whatsbox/}


\subsection{Effizientes und verantwortungsvolles
Datenmanagement im
Zeitalter der DSGVO}
(KL)\\

Quelle: \cite{Brockmann2018}
Nach DSGVO müssen Unternehmen jederzeit: 
\begin{itemize}
\item	darlegen können, welche personenbezogenen Daten in welchen
Systemen für welchen Zweck auf welcher rechtlichen Basis von
wem gespeichert und verarbeitet werden,
\item darlegen können, wer diese Daten wie und wofür verarbeitet
und nutzt,
\item Anfragen binnen eines Monats vollständig, lesbar und eindeutig
beantworten,
\item Änderungs-, Widerruf- oder Löschungswünsche seitens der
betroffenen Person auf allen Systemen im Unternehmen umsetzen,
\item sicherstellen, dass (auch externe) Datenverarbeiter der DSGVO
nachkommen und
\item die Konformität mit den datenschutzrechtlichen Anforderungen
prüfen, nachbessern und nachweisen.

\end{itemize}

in Unternehmen herrschen in der Regel heterogene Datenlandschaften vor, die eine DSGVO-konformität erschweren.

Der Artikel versucht dem Problem durch semantische Technologien zu begegnen (Ontologien/Vokabularen). Bsp Google Knowledgegraph.
Die semantische Technologie kann als Enterprise Knowledge
Graph (EKG) auch auf Unternehmen und Institutionen angewendet
werden.
Kern der EKG ist das standardisierte, Statement-zentrische
RDF-Datenmodell. Dieses Modell strukturiert Daten nach einer
Subjekt-Prädikat-Objekt-Syntax, wodurch Daten jeglicher Art
untereinander referenziert und verlinkt werden können.
Bei einigen EKG-Ansätzen wird die Transformation aller Unternehmensdaten
in RDF empfohlen. Dies hat in der Vergangenheit
jedoch zu Problemen bei der Skalierbarkeit geführt, da alle
Daten transformiert werden mussten. Als Alternative hat sich
deshalb das rein Metadaten-basierte Management der Quelldaten
mittels eines semantischen Datenkatalogs herausgebildet. Hierbei
bleiben die Quelldaten in ihren Applikationen unberührt,
und nur ihre Metadaten (im RDF-Format) werden in einem zentralen
Datenkatalog zusammengeführt und mit der Quelle verlinkt.
Insbesondere für die Umsetzung der DSGVO ist die Katalogisierung,
Beschreibung und Vernetzung von persönlichen Daten
aus einer Vielzahl von Quellen eine essentielle Voraussetzung.
In der Praxis wird in die bestehende Datenlandschaft eines
Unternehmens ein semantischer Datenkatalog integriert. In diesem
werden die Metadaten, als das Wissen über die im Unternehmen
vorliegenden Daten zusammengeführt und gespeichert.
Die Quelldaten in den verteilten Applikationen im Unternehmen
werden dabei nicht berührt.

	\include{Mitarbeiter}

	\chapter{Einzelne Publikationen}
	
	\chapter{Paper: Bounded Privacy}
(Von M. Alpers)\\

Zusammenfassung der Veröffentlichung \emph{Bounded Privacy: Formalising the Trade-Off Between Privacy and Quality of Service} aus \emph{Sicherheit 2018, 23}\\

\section{Relevanz}

Potentiell relevant:\\

Es wird das Verhältnis zwischen Datenschutz und Notwendigkeit der Offenlegung von geschützten Daten untersucht, damit Kategorien von Anwendungen nutzbar sind. Ggf. lassen sich aber aus dieser Arbeit Schutzziele für Unternehmen expolieren, denn die Mitarbeiter von Unternehmen sind ja ihrerseits Nutzer von Anwendungen. Damit ist die Unternehmenssicherheit hier ggf. mittelbar betroffen.
	\section{Usability von Security-APIs}
(Von M. Alpers)\\

Zusammenfassung der Veröffentlichung \emph{Usability von Security-APIs für massiv-skalierbare vernetzte Service-orientierte System} aus \emph{Lecture Notes in Informatics, 2018, S. 285, ff.}\\

\subsection{Relevanz}

Relevant:\\

In diesem Artikel wird die Notwendigkeit von Security-APIs bei der Entwicklung von Service-orientierten Systemen hervorgehoben und auf die Schwierigkeit, diese in ein heterogenes Entwicklerteam einzuführen. Da die Realisierung von Sicherheitsstandards direkt die IT-Sicherheit eines Unternehmens betrifft, sollten sich einzelne Aussagen für unsere Arbeit ableiten lassen.
	\chapter{Paper: Fallstricke Inhaltsanalyse Mails}
(Von M. Alpers)\\

Zusammenfassung der Veröffentlichung \emph{Fallstricke bei der Inhaltsanalyse von Mails: Beispiele, Ursachen und Lösungsmöglichkeiten} aus \emph{Sicherheit 2018, S. 253, ff.}\\

\section{Relevanz}

Sehr relevant:\\

In diesem Artikel wird E-Mail als Angriffsmittel \glqq{}zur Infektion mit Malware und zum Phishing\grqq{} aufgeführt. Es wird erläutert, dass durch den \emph{MIME}-Standard die Sicherheitsanalyse erschwert wird. Weiter wird auf Methoden verwiesen, mit denen hierauf reagiert werden kann.
	\include{ix0918Einseitig}
	
	\chapter{Relevante Institutionen und Gruppen}
	
	\chapter{Institutionen und Gruppen}
(Von M. Alpers)\\

Die folgenden Institutionen und Gruppen können hilfreiche Informationen zum Thema unserer Arbeit beitragen:

\begin{itemize}
	\item BSI
\end{itemize}

\section{BSI - Bundesamt für Sicherheit in der Informationstechnik}\index{BSI}\index{Bundesamt für Sicherheit in der Informationstechnik}

Das BSI befasst sich laut eigener Angabe \glqq{}mit allen Fragen rund um die IT-Sicherheit in der Informationsgesellschaft.\grqq{} (Quelle: \url{https://www.bsi.bund.de/DE/Service/FAQ/faq_node.html})\\

Es bietet folgende für diese Arbeit relevanten Dienste:\\
(Quelle: \url{https://www.bsi.bund.de/DE/Service/FAQ/faq_node.html})

\begin{itemize}
	\item \glqq{}Prüfung, Zertifizierung und Akkreditierung von IT-Produkten und -Dienstleistungen\grqq{}
	\item \glqq{}Warnung vor Schadprogrammen oder Sicherheitslücken in IT-Produkten und -Dienstleistungen\grqq{}
	\item \glqq{}Entwicklung einheitlicher und verbindlicher IT-Sicherheitsstandards\grqq{}
\end{itemize}

\subsection{Cybersicherheitsstrategie 2016}\index{Cybersicherheitsstrategie}

Das Bundesministerium des Inneren hat als Grundlage staatlicher Aktivitäten im Bereich IT-Sicherheit die sog. \emph{Cybersicherheitsstrategie} veröffentlicht. (Letzte Aktualisierung: 2016) Darin wird es als Aufgabe des Bundes definiert, \glqq{}Unternehmen in Deutschland zu schützen\grqq{} (Quelle: \url{https://www.bmi.bund.de/SharedDocs/downloads/DE/publikationen/themen/it-digitalpolitik/cybersicherheitsstrategie-2016.pdf} : \emph{Handlungsfeld 2} (S. 10)) Die konkrete Umsetzung dieses Ziels wird jedoch nicht ausgeführt, sondern lediglich die Absicht bekundet, stärken mit entsprechenden Ansprechstellen zusammen zu arbeiten. (Gleiche Quelle, S. 22)\\

\subsection{Überblickspapiere des BSI}

Unter der Bezeichnung \emph{Überblickspapiere}\index{Überblickspapiere} veröffentlicht das BSI Informationen über spezifische Sicherheitsbedrohungen bei der Nutzung bestimmter Systeme. Es liegen zurzeit die folgenden Überblickspapiere vor:\\
(Quelle: \url{https://www.bsi.bund.de/DE/Themen/ITGrundschutz/Ueberblickspapiere/Ueberblickspapiere_node.html})

\begin{itemize}
	\item Android (Stand: 20.11.2014, Android v 4.4.4)
	\item Apple iOS (Stand: 24.07.2013, iOS v 6)
	\item Online-Speicher (Stand: 08.11.2012)
	\item Smartphones (Stand: Unbekannt)
	\item Netzzugangskontrolle (Stand: Unbekannt)\\
	Hier wird der Umgang mit fremden mobilen Geräten in einer geschützten Umgebung betrachtet.
	\item IT-Consumerisation und BYOD (Stand: 31.07.13)\\
	Hier wird der Umgang mit mobilen Endgeräten innerhalb eines Unternehmens betrachtet, die von Mitarbeitern mitgebracht werden.
\end{itemize}

\subsubsection{Fazit:}

\begin{itemize}
	\item Die Überblickspapiere zu Android und iOS sind mit vier bzw. fünf Jahren veraltet, da sie vorrangig auf die konkreten Gefahren der jeweiligen OS-Version eingehen.
	\item Das Überblickpapier Consumerisation und BYOD ist zwar mehr als vier Jahre alt, aber allgemeiner formuliert, sodass es sich als Grundlage einer Sicherheitsstrategie für ein Unternehmen nutzen lässt.
	\item Gleiches gilt für das Überblickpapier Online-Speicher (Cloud), Smartphones
	\item Das Überblickpapier Netzzugangskontrolle ist nicht datiert, geht zum Teil auf konkrete Standards aber auch auf grundlegende Sicherheitszenarien ein. Es sollte damit zumindest als Grundlage für den Entwurf eines Sicherheitskonzepts taugen.
\end{itemize}
	\include{AllianzFuerCybersicherheit}
	
	
	%%% Appendices %%%
	
	\renewcommand{\indexname}{Stichwortverzeichnis}		% Legt den Titel des Stichwortverzeichnisses fest.
	\addcontentsline{toc}{chapter}{Stichwortverzeichnis}
	\printindex
	\nocite{*}
	\bibliography{lit}

\end{document}
