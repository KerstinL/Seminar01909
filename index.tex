\documentclass[11pt, a4paper, twoside]{book}

\usepackage{palatino, url}
\usepackage[ngermanb]{babel}
\usepackage[utf8]{inputenc}
\setlength{\parindent}{0cm}

\usepackage{amsfonts}	% für mathematische Symbole wie Menge der nat. Zahlen
\usepackage{amsmath}
\usepackage{textcomp}	% für \textmu also das Mü- bzw. Mikro-Symbol
\usepackage{makeidx}	% für ein "schöneres" Stichwortverzeichnis
\makeindex				% erzeugt das Stichwortvereichnis, blendet es jedoch nicht hier ein. Dafür am Ende den entspr. Code verwenden.

\bibliographystyle{plain}

\begin{document}
	
	%%% Title declaration begin %%%
	
	\title{Maßnahmen zur Absicherung von privaten und kleinen Unternehmensnetzwerken}
	\author{B.Sc. Markus Alpers, 6872530 \\ Kerstin Lapp, 515200}
	\date{\today}
	
	\maketitle
	
	%%% Title declaration end %%%
	
	\tableofcontents
	
	%%% Chapters %%%
	
	\part{Grundlagen der IT-Sicherheit (K. Lapp)}
	
	\section*{Schutzziele und Bedrohungen}
(von K. Lapp)

\subsection{Angriffsziele} (nach 1866 KE1 S.11):

\begin{itemize}
  \item Kommunikationswege
  \item Computer
  \item Daten
\end{itemize}
Schutzziele (nach 186 KE1 S.12-16):

\begin{itemize}
  \item Vertraulichkeit: Daten sind nur befugten Personen zugänglich. \\
  Bedrohung: unbefugter Informationsgewinn.
  \item Integrität: Daten sind korrekt und unverändert. \\ Bedrohung: Unbefugte Modifikation
  \item Authentizität: Daten stammen von vorgeblichen Erzeuger. \\
  Bedrohung: unbefugte Erzeugung
  \item Verfügbarkeit: Daten können von befugten Personen gelesen/bearbeitet werden. \\
  	Bedrohung: unbefugte Unterbrechung
\end{itemize}


	\section*{VLAN}
(von K. Lapp)\\

VLANs = Virtuelle Netze \\
\cite[S.167-169]{zisler2018computer} \\
\cite{bsiLogSeg}\\

VLANs dienen der logischen Segmentierung von Netzen. Es sind logische Teilnetze, die an Switches gebildet werden. Es können Gruppen gebildet werden, ohne dass in die physische Vernetzung eingegriffen wird.
Gründe für den VLAN Einsatz nach \cite[S.167]{zisler2018computer}:
\begin{itemize}
  \item Eindämmung von Broadcast durch mehrere Broadcast Domänen
  \item Abbildung der betrieblichen Organisationsstruktur (Abteilungen)
  \item Einteilung des Netzes nach Anwendung 
\end{itemize}



Arten von VLANs: 
\begin{enumerate}
  \item Statische VLANs = Portbasierte VLANs: Switch-Ports werden fest einem VLAN zugeordnet. Port kann nur einem VLAN zugeordnet werden. 
  \item Dynamisches VLAN = Paketbasiertes VLAN, auch tagged VLAN: Ein Port kann mehreren VLANs angehören. Pakete werden gekennzeichnet welchem VLAN sie angehören. Achtung hohe Manipulationsgefahr. 
\end{enumerate}


	\section{Virtuelle Private Netzwerke VPN}
	\section*{Firewall/DMZ}
(von K. Lapp)\\

Kurs 1866 KE4  S199: 

Screened Subnet/DMZ: Eigenes Teilnetz zwischen Intranet und Internet, welches an beiden Eden durch Paketfilter geschützt wird. \\
Architektur: Internet-Router-ALG-Rechner-Router-Intranet.\\
Router auf Seite des Internet lässt nur Pakete zum Rechner innerhalb der DMZ durch. Router zum Intranet lässt nur Pakete vom Rechner aus der DMZ durch. Die andere Richtung analog. Angreifer muss also mehrere Systeme angreifen ehe er das Intranet erreichen kann. ALG filtert zusätzlich Kommunikation auf der Anwendungsebene. Bei erhöhtem Arbeitsaufkommen sind verschiedene ALG für verschiedene Anwendungen möglich (ftp, telnet, http). Öffentliche Webserver sollten sich auch in der DMZ befinden.\\
DMZ kann auch mit einem Router mit 3 Anschlüssen verwirklicht werden: \begin{enumerate}
  \item Internet
  \item DMZ
  \item Intranet
\end{enumerate}
mit entsprechend konfiguriertem Paketfilter.

	\section*{Sicherheitsleitlinie/Polocies}
Siehe 1866 KE4 Kapitel 4.4 ab S.209 und BSI Grundschutzkatalog

Es müssen Richtlinien existieren WER fürdie Sicherheit verantwortlich ist (\cite{lipp2007vpn}).
	\section*{Datenschutz und Jugendschutz}

\subsection{Blog von Microsoft: Forderung nach Regulierung von Gesichtserkennung}
(von M. Alpers, muss noch ausgearbeitet werden)\\

Unter \url{https://blogs.microsoft.com/on-the-issues/2018/07/13/facial-recognition-technology-the-need-for-public-regulation-and-corporate-responsibility/} findet sich ein Blog-Eintrag des Microsoft-Präsidenten Brad Smith (weitere Blogeinträge von ihm unter: \url{https://blogs.microsoft.com/on-the-issues/author/bradsmith/} ), in dem dieser Politik und Wirtschaft zum verantwortlichen Umgang mit Gesichtserkennung aufruft.

\subsection{WhatsApp datenschutzkonform nutzen}
(von M. Alpers)\\

Ein zentrales Problem für den Datenschutz (auch) in Unternehmen sind Funktionen zum \emph{Data Mining}\index{Data Mining} in Anwendungen, die zum Teil als sog. \emph{Bloatware}\index{Bloatware} installiert werden. Deren Geschäftsmodell basiert dann nicht darauf, dass der Nutzer eine Lizenzgebühr zahlt, sondern durch die Offenlegung eines Teils seiner Daten die Grundlage für das Wirtschaftmodell des Unternehmens bildet. Im Falle von WhatsApp sind die offengelegten Daten unter anderen die im Smartphone gespeicherten Kontakte des Nutzers. Dies widerspricht der DSVGO, da diese ein ausdrückliches Einverständnis der Betroffenen fordert.\\

\glqq{}Das Unternhemen  Backes SRT hat eine App namens \emph{WhatsBox} veröffentlicht,\grqq{} die nun verspricht, WhatsApp in einer Sandbox zu betreiben und dadurch WhatsApp DSVGO-konform nutzbar zu machen. Über die lassen sich Kontakte explizit für WhatsApp freigeben.\\

\url{https://www.backes-srt.com/de/loesungen/whatsbox/}
	\include{Mitarbeiter}
	
	\part{IT-Sicherheit in kleinen bis mittleren Unternehmen (M. Alpers)}

	\include{TeilVonMarkus}
	
	%%% Appendices %%%
	
	\renewcommand{\indexname}{Stichwortverzeichnis}		% Legt den Titel des Stichwortverzeichnisses fest.
	\addcontentsline{toc}{chapter}{Stichwortverzeichnis}
	\printindex
	\nocite{*}
	\bibliography{lit}

\end{document}
