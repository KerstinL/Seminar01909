\chapter{Paper: Bounded Privacy}
(Von M. Alpers)\\

Zusammenfassung der Veröffentlichung \emph{Bounded Privacy: Formalising the Trade-Off Between Privacy and Quality of Service} aus \emph{Sicherheit 2018, 23}\\

\section{Relevanz}

Potentiell relevant:\\

Es wird das Verhältnis zwischen Datenschutz und Notwendigkeit der Offenlegung von geschützten Daten untersucht, damit Kategorien von Anwendungen nutzbar sind. Ggf. lassen sich aber aus dieser Arbeit Schutzziele für Unternehmen expolieren, denn die Mitarbeiter von Unternehmen sind ja ihrerseits Nutzer von Anwendungen. Damit ist die Unternehmenssicherheit hier ggf. mittelbar betroffen.